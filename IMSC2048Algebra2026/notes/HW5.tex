\documentclass{article}
%\usepackage[UTF8]{ctex}

\usepackage{tikz-cd}

\usepackage{amsmath}
\usepackage{amssymb}
\usepackage{amsthm}
\newtheorem{ex}{Excercise}
\newtheorem{defn}{Definition}
\newtheorem{thm}{Theorem}

\newcommand{\C}{\mathbb{C}}
\newcommand{\Z}{\mathbb{Z}}
\newcommand{\R}{\mathbb{R}}
\newcommand{\Q}{\mathbb{Q}}
\newcommand{\F}{\mathbb{F}}


\newcommand{\tr}{\operatorname{tr}}
\newcommand{\GL}{\operatorname{GL}}
\newcommand{\SL}{\operatorname{SL}}
\newcommand{\rank}{\operatorname{rank}}
\newcommand{\im}{\operatorname{im}}
\newcommand{\Aut}{\operatorname{Aut}}
\newcommand{\SO}{\operatorname{SO}}
\newcommand{\Hom}{\operatorname{Hom}}

\title{IMSC 2048\ HW5 \\ Due 2026/2/12}


\begin{document}
\maketitle


\section{Excercies}

\subsection{Mandatory part}

\begin{ex}
  Let $V$ be an irreducible representation of a finite group $G$ over $\C$. Assume $V$ is not the trivial representation. Prove that for any $v\in V$, we have $\sum_{g \in G} g \cdot v = 0$. (For cyclic groups, this is known to be an identity of roots of unity.)
\end{ex}

\begin{ex}
  (Artin Algebra Chapter 10, 3.5) Let $x$ be a generator of a cyclic group $G$ of order $p$. Sending 
    \[
    x \mapsto \begin{pmatrix} 1 & 1 \\ 0 & 1 \end{pmatrix}
    \]
    defines a matrix representation $G \to \GL_2(\mathbb{F}_p)$. Prove that this representation is not the direct sum of irreducible representations.
\end{ex}

\begin{ex}
  (Artin Algebra Chapter 10, 3.4)
    Let $\langle \cdot, \cdot \rangle$ be a nondegenerate skew-symmetric form on a vector space $V$, and let $\rho$ be a representation of a finite group $G$ on $V$. Prove that the averaging process produces a $G$-invariant skew-symmetric form on $V$, and show by example that the form obtained in this way need not be nondegenerate.
\end{ex}

\begin{ex}
  (Artin algebra Chapter 10, 2.2) Consider the standard two-dimensional complex representation of the dihedral group $D_n$. For which $n$ is this an irreducible complex representation?

  Here the standard representation is given by the action of $D_n$ as the group of symmetries of a regular $n$-gon in the plane, or equivalently, the representation defined by the matrices
  \[r \mapsto \begin{pmatrix} \cos(2\pi/n) & -\sin(2\pi/n) \\ \sin(2\pi/n) & \cos(2\pi/n) \end{pmatrix}, \quad s \mapsto \begin{pmatrix} 1 & 0 \\ 0 & -1 \end{pmatrix},\]
  where $r$ is a rotation by $2\pi/n$ and $s$ is a reflection.
\end{ex}

\begin{ex}
  (Artin Algebra Chapter 10, 3.1) Let $G$ be a cyclic group of order $3$. The matrix
  \[A=\begin{pmatrix} 0 & 1 \\ -1 & -1 \end{pmatrix}\]
   has order $3$, so it defines a matrix representation of $G$ on $\mathbb{C}^2$. Use the averaging process to produce a $G$-invariant form from the standard Hermitian product $\langle X, Y \rangle = X^*Y$ on $\mathbb{C}^2$.
\end{ex}

\begin{ex}
  (Artin Algebra Chapter 10, 3.2) Let $\rho\colon G \to \GL(V)$ be a representation of a finite group $G$ on a real finite-dimensional vector space $V$. Prove the following:
  \begin{enumerate}
    \item There exists a $G$-invariant, positive definite symmetric form $\langle \cdot, \cdot \rangle$ on $V$.
    \item $\rho$ is a direct sum of irreducible representations.
    \item Every finite subgroup of $\GL_n(\mathbb{R})$ is conjugate to a subgroup of $\mathrm{O}(n)$.
  \end{enumerate}
\end{ex}



%\begin{ex}
%  (Artin Algebra Chapter 10, 7.1) Prove a converse to Schur's Lemma: Let $\rho$ be a complex representation of a finite group $G$ on a vector space $V$. If the only $G$-invariant linear operators on $V$ are scalar multiplications (i.e., $\Hom_G(V, V) = \mathbb{C} \cdot \id_V$), then $\rho$ is irreducible.
%\end{ex}

\begin{ex}
  Show that the dual representation of an irreducible representation is also irreducible.
\end{ex}




\subsection{Optional excercises}
\begin{ex}
  In this part, we prove the semisimplicity theorem (Maschke's theorem) for any representation of group $G$ and field $F$ such that the characteristic of $F$ does not divide the order of $G$, using the averaging technique.

  Let $V$ be a representation of $G$ over $F$, and let $W$ be a $G$-invariant subspace of $V$. Prove that there exists a complementary $G$-invariant subspace $W'$ of $V$ such that $V = W \oplus W'$. (Hint: start with any complementary subspace and then use averaging to construct a projection onto $W$ whose kernel is $W'$.)
\end{ex}


\begin{ex}
Show that the dual representation is isomorphic to the original representation for any finite group representation if and only if there exists a nondegenerate $G$-invariant bilinear form on the representation space.
\end{ex}



\end{document}