\documentclass{article}
%\usepackage[UTF8]{ctex}

\usepackage{tikz-cd}

\usepackage{amsmath}
\usepackage{amssymb}
\usepackage{amsthm}
\newtheorem{ex}{Excercise}
\newtheorem{defn}{Definition}
\newtheorem{thm}{Theorem}

\newcommand{\C}{\mathbb{C}}
\newcommand{\Z}{\mathbb{Z}}
\newcommand{\R}{\mathbb{R}}
\newcommand{\Q}{\mathbb{Q}}
\newcommand{\F}{\mathbb{F}}


\newcommand{\tr}{\operatorname{tr}}
\newcommand{\GL}{\operatorname{GL}}
\newcommand{\SL}{\operatorname{SL}}
\newcommand{\rank}{\operatorname{rank}}
\newcommand{\im}{\operatorname{im}}
\newcommand{\Aut}{\operatorname{Aut}}
\newcommand{\SO}{\operatorname{SO}}
\newcommand{\Hom}{\operatorname{Hom}}
\newcommand{\id}{\operatorname{Id}}

\title{IMSC 2048\ HW7 \\ Due 2026/3/5}


\begin{document}
\maketitle


\section{Exercises}
\subsection{Mandatory part}

In this homework, we always work over the complex field $\C$ and finite dimensional representations of finite groups.
\begin{ex}
  You can skip the first three questions if you already learned them in group theory last semester. Let $[G,G]$ be subgroup of $G$ generated by all commutators $[g,h]=ghg^{-1}h^{-1}$ for $g,h\in G$. 
  \begin{enumerate}
    \item Show that $[G,G]$ is a normal subgroup of $G$.
    \item Show that $G/[G,G]$ is abelian.
    \item Show that if $H$ is abelian and $\phi\colon G \to H$ is a group homomorphism, then $[G,G]\subset \ker \phi$.
    \item Prove that there is an one-to-one correspondence between the one-dimensional representations of $G$ and the irreducible representations of $G/[G,G]$.
    \item Show that the commutator subgroup $[G,G]$ is the intersection of kernels of all one-dimensional characters of $G$.
  \end{enumerate}
\end{ex}

\begin{ex}
  From class, we know that the character $\chi_{reg}$ of regular representation satisfies $\chi_{reg}(g)=0$ if $g\neq e$. There is an inverse of this proposition. Let $\chi$ be a character of $G$ and satisfies $\chi(g)=0$ if $g\neq e$. Prove that the corresponding representation is the direct sum of several copies of regular representation, i.e. $\rho\cong \rho_{reg}\oplus \rho_{reg}\cdots \oplus \rho_{reg}$ for some integer $n$. 
\end{ex}

\begin{ex}
  In this question, you will find the character table of $A_4$ the group of even permutations of $S_4$. Alternating group $A_4$ is a subgroup of $S_4$ and has 4 conjugacy classes $$\{(1)\}, \{(12)(34), (23)(14), (13)(24)\}, \{(123),(142), (134), (243)\}, \{(132), (124), (143), (234)\}.$$ 
\begin{enumerate}
\item Prove that $K=\{(1), (12)(34), (23)(14), (13)(24)\}$ is a normal subgroup of $S_4$.
\item Prove that $A_4/K$ is cyclic group of order $3$. (You can use the fact that any group of prime order is a cyclic group, think about why this is true.)
\item Use lifting to find all the irreducible characters of $A_4$.
\item (Optional) Compare this with character table of $S_4$ and describe all the irreducible representations of $A_4$.
\end{enumerate}
\end{ex}

\begin{ex}
Let $G$ be a finite group and $g\in G$. Prove that $g$ and $g^{-1}$ are in the same conjugacy class if and only if $\chi(g)$ is in $\R$ for all characters $\chi$. 
\end{ex}

\begin{ex}
  Below is a partial character table for $G$. One conjugacy class and one row are missing. Here $e$ is the identity element. The numbers in the top row mean the numbers of elements in each conjugacy class.

 \begin{equation}
\begin{array}{l|ccccc} 
& (1) & (1) & (2) & (2) & (3) \\
& e & u & v & w & x \\
\hline \chi_{1} & 1 & 1 & 1 & 1 & 1 \\
\chi_{2} & 1 & 1 & 1 & 1 & -1 \\
\chi_{3} & 1 & -1 & 1 & -1 & \sqrt{-1} \\
\chi_{4} & 1 & -1 & 1 & -1 & -\sqrt{-1} \\
\chi_{5} & 2 & 2 & -1 & -1 & 0
\end{array}
\end{equation}
\begin{enumerate}
\item Complete the character table. (Hint: the number of elements in each conjugacy class divides the order of $G$.)

\item Determine the normal subgroups of $G$.

\item Find the orders of representative elements in each conjugacy class.
\end{enumerate}
\end{ex}

\begin{ex}
  Let $\C[G]$ be the underlying vector space for regular representation, i.e. the elements in $\C[G]$ are formal linear combinations of elements in $G$. Let $V$ be a group representation of $G$. Prove that the map $F\colon Hom_G(\C[G], V) \to V$ defined by $F(f)=f(e)$ gives a linear space isomorphism. Use this isomorphism to give a new proof of the fact that the multiplicity of an irreducible representation $W$ in regular representation is equal to $\dim W$.
\end{ex}




\subsection{Optional problems}

\begin{ex}
  Write down the character table for group $A_5$, the group of even permutations on $5$ elements. (Hint: $A_5$ has $5$ conjugacy classes and thus has $5$ irreducible representations.)
\end{ex}



\end{document}