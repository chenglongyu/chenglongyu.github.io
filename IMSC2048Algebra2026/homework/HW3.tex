\documentclass{article}
%\usepackage[UTF8]{ctex}

\usepackage{tikz-cd}

\usepackage{amsmath}
\usepackage{amssymb}
\usepackage{amsthm}
\newtheorem{ex}{Excercise}

\newcommand{\C}{\mathbb{C}}
\newcommand{\Z}{\mathbb{Z}}
\newcommand{\R}{\mathbb{R}}
\newcommand{\Q}{\mathbb{Q}}
\newcommand{\F}{\mathbb{F}}


\newcommand{\tr}{\operatorname{tr}}
\newcommand{\GL}{\operatorname{GL}}
\newcommand{\SL}{\operatorname{SL}}
\newcommand{\rank}{\operatorname{rank}}
\newcommand{\im}{\operatorname{im}}
\newcommand{\Aut}{\operatorname{Aut}}

\title{IMSC 2048\ HW3 \\ Due 2026/1/29}


\begin{document}
\maketitle


\section{Excercises}

\subsection{Useful Exercises}

\begin{ex}
  Suppose $A$ is an invertible real square matrix with singular values $\sigma_1, \cdots, \sigma_n$. Find the singular values of $A^{-1}$.
\end{ex}


\begin{ex}[Polar Decomposition]
In this problem, you will prove the polar decomposition of an invertible complex matrix using the singular value decomposition.
  
\begin{enumerate}
\item Try to state without proof the singular value decomposition theorem for complex matrices similar to the one stated in the class for real matrices. 
\item Use it to prove the following: Let $A$ be an $n\times n$ invertible complex matrix. Prove that there exist unitary matrices $U\in \mathrm{U}(n)$ and $P$ is a positive definite hermitian matrix such that $A=UP$. This is called the polar decomposition of $A$.
\item Show that the polar decomposition is unique. (This shows that $\GL(n,\C)$ is homeomorphic to $\mathrm{U}(n)\times \mathcal{H}_n^{+}$ where $\mathcal{H}_n^{+}$ is the set of all positive definite hermitian matrices, a convex cone in the vector space of all hermitian matrices.)
\end{enumerate}
\end{ex}


\begin{ex}
Artin Chapter 8, Excercise 6.20

Prove the circulant, the matrix below, is normal.
$$C=\begin{bmatrix}
c_0 & c_{1} & c_{2} & \cdots & c_{n}\\
c_{n} & c_0 & c_{1} & \cdots & c_{n-1}\\
c_{n-1} & c_{n} & c_0 & \cdots & c_{n-2}\\
\vdots & \vdots & \vdots & \ddots & \vdots\\
c_{1} & c_{2} & c_{3} & \cdots & c_0
\end{bmatrix}$$

(How to diagonalize it? Hint: write $C$ as a polynomial in the shift matrix. see Artin Chapter 8, Excercise 6.19)
\end{ex}

\begin{ex}
Prove that for any square submatrix of a unitary matrix, the modulus of any of its complex eigenvalues does not exceed $1$.
\end{ex}

\begin{ex}
\begin{enumerate}
     \item If $A$ and $B$ are normal matrices, is $AB$ necessarily normal? What if we additionally assume $AB=BA$?
      
      \item Determine if the matrix $A=\begin{pmatrix}
        \sqrt{-1} & -\sqrt{-1}\\
        -\sqrt{-1} & \sqrt{-1}
      \end{pmatrix}$ is normal, Hermitian, or unitary.
\end{enumerate}  
\end{ex}


\begin{ex}[Characterization of Normal Matrices in terms of Singular Values]
Let $A$ be an $n \times n$ complex matrix. Let $\sigma_i(A)$ denote the $i$-th singular value of $A$ (such that $\sigma_1(A)\ge \cdots \ge \sigma_n(A)\ge 0$), and let $\lambda_1, \cdots, \lambda_n$ be the complex eigenvalues of $A$ (counted with algebraic multiplicity).
Prove that $\sum\limits_{i=1}^n\sigma_i(A)^2\ge \sum\limits_{i=1}^n |\lambda_i|^2$, with equality if and only if $A$ is a normal matrix.

(Hint: Recall that any complex matrix can be conjugate to upper triangular matrix by an invertible matrix, by choosing eigenvectors and induction. Prove that actually a unitary matrix can do this job, then compare the Frobenius norms of both sides.)
\end{ex}

\begin{ex}
Lang Algebra Chapter XV, 1.

Here we choose $\sigma$ to be complex conjugation.
\begin{enumerate}
	\item 
 Let $E$ be a finite dimensional space over the complex numbers, and let
$$
h: E \times E \rightarrow \mathbb{C}
$$
be a hermitian form. Write
$$
h(x, y)=g(x, y)+i f(x, y)
$$
where $g, f$ are real valued. Show that $g, f$ are $\mathbb{R}$-bilinear, $g$ is symmetric, $f$ is alternating.
\item  Let $E$ be finite dimensional over $\mathbf{C}$. Let $g: E \times E \rightarrow \mathbf{C}$ be $\R$-bilinear. Assume that for all $x \in E$, the map $y \mapsto g(x, y)$ is $\mathbb{C}$-linear, and that the $\R$-bilinear form
$$
f(x, y)=g(x, y)-g(y, x)
$$
is real-valued on $E \times E$. Show that there exists a hermitian form $h$ on $E$ and a symmetric $\C$-bilinear form $\psi$ on $E$ such that $2 i g=h+\psi$. Show that $h$ and $\psi$ are uniquely determined.
\end{enumerate}
\end{ex}

\begin{ex}
    Let $X=A+\sqrt{-1}B$ be a complex square matrix, where $A,B\in M_n(\R)$. Prove that $X$ is a unitary matrix if and only if
    $$\begin{pmatrix}
      A & -B\\
      B & A
        \end{pmatrix}$$
     is an orthogonal matrix.
\end{ex}


  

\subsection{Optional problems}

\begin{ex}
	Prove the additive inequality of singular values. Let $A, B$ be two $m\times n$ real matrices. Prove that
	$$
	\sigma_{k+l-1}(A+B)\leq \sigma_k(A)+\sigma_l(B)
	$$
	for $k+l-1\leq \min(m,n)$.
\end{ex}

\begin{ex}
  Let $(V,\omega)$ be a symplectic space if $\omega$ is a non-degenerate skew-symmetric form on the $F$-vector space $V$. An $F$-linear transformation $T$ is called a symplectic transformation if $\omega(T(v), T(w))=\omega(v, w)$. A basis $(\alpha_1, \cdots, \alpha_n, \beta_1, \cdots, \beta_n)$ is called a symplectic basis if $\omega(\alpha_i, \beta_j)=\delta_{ij}$ and $\omega(\alpha_i, \alpha_j)=\omega(\beta_i, \beta_j)=0$. When $F=\C$, prove that for any symplectic transformation $T$ of a symplectic space, there exists a symplectic basis such that the matrix of $T$ under this basis has the form
  $$\begin{bmatrix}
    B_n & 0_n\\
    0_n & B_n^T
  \end{bmatrix}$$
  where $B_n$ is an $n$-dimensional Jordan normal form.
\end{ex}


\begin{ex}
  Prove the following inequality between singular values and eigenvalues. Let $A$ be an $n\times n$ complex matrix with eigenvalues $\lambda_1, \cdots, \lambda_n$ (counted with algebraic multiplicity) and singular values $\sigma_1\ge \sigma_2\ge \cdots \ge \sigma_n\ge 0$. Then for any $k=1, \cdots, n$,
  $$|\lambda_1\lambda_2\cdots \lambda_k|\le \sigma_1\sigma_2\cdots \sigma_k.$$
\end{ex}

\begin{ex}[Challenge Problem]
  Let $A$ be a real square matrix of order $n$, and let the eigenvalues of $A^T A$ be $\lambda_1^2, \cdots, \lambda_n^2$, where $0 \leq \lambda_i \leq 1$ for $i=1, 2, \cdots, n$. Prove that:

$$
\det(I_n-A) \geq\left(1-\lambda_1\right)\left(1-\lambda_2\right) \cdots\left(1-\lambda_n\right)
$$ 
\end{ex}

\end{document}