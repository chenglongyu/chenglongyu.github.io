\documentclass{article}
%\usepackage[UTF8]{ctex}

\usepackage{tikz-cd}

\usepackage{amsmath}
\usepackage{amssymb}
\usepackage{amsthm}
\newtheorem{ex}{Excercise}
\newtheorem{defn}{Definition}
\newtheorem{thm}{Theorem}

\newcommand{\C}{\mathbb{C}}
\newcommand{\Z}{\mathbb{Z}}
\newcommand{\R}{\mathbb{R}}
\newcommand{\Q}{\mathbb{Q}}
\newcommand{\F}{\mathbb{F}}


\newcommand{\tr}{\operatorname{tr}}
\newcommand{\GL}{\operatorname{GL}}
\newcommand{\SL}{\operatorname{SL}}
\newcommand{\rank}{\operatorname{rank}}
\newcommand{\im}{\operatorname{im}}
\newcommand{\Aut}{\operatorname{Aut}}
\newcommand{\SO}{\operatorname{SO}}

\title{IMSC 2048\ HW4 \\ Due 2026/2/5}


\begin{document}
\maketitle


\section{Excercies}

\subsection{Mandatory part}

\begin{ex}
  Is it true that the conjugacy classes of unitary group \(\mathrm{U}(2)\) are determined by the trace and determinant? Prove your answer.
\end{ex}


\begin{ex}
\begin{defn}[Orthogonal Group of Signature \((p, q)\)]
  Let \(p, q\) be non-negative integers. The orthogonal group of signature \((p, q)\), denoted by \(\mathrm{O}_{p, q}\), is defined as the group of all linear transformations on \(\mathbb{R}^{p+q}\) that preserve the bilinear form
  $$
  \langle x, y\rangle = x_1 y_1 + x_2 y_2 + \cdots + x_p y_p - x_{p+1} y_{p+1} - \cdots - x_{p+q} y_{p+q},
  $$
  for all \(x, y \in \mathbb{R}^{p+q}\).
  
\end{defn}

  Let \(W\) be the space of real trace-zero \(2 \times 2\) matrices $W=\{A\in M_{2\times 2}(\R)| trace(A)=0\}$. $W$ has a basis \(\mathbf{B}=\left(w_{1}, w_{2}, w_{3}\right),\) where
$$
w_{1}=\left[\begin{array}{cc}
{1} & {} \\
{} & {-1}
\end{array}\right], w_{2}=\left[\begin{array}{cc}
{0}&{1} \\
{1}&{0}
\end{array}\right], w_{3}=\left[\begin{array}{cc}
{0}&{1} \\
{-1}&{0}
\end{array}\right]
$$
\begin{enumerate} 
\item
Show that the symmetric bilinear form defined by \(\langle A, A^\prime \rangle=\operatorname{trace}\left(A A^{\prime}\right)\) has signature $(2, 1)$. (Hint: use basis \(\mathbf{B}\))
\item Prove that $P\star A= PAP^{-1}$ defines a linear group operation of \(\SL({2}, \R)\) on the space \(W .\) 
\item Use this operation to define a group homomorphism \(\varphi: \SL({2, \R}) \rightarrow \mathrm{O}_{2,1}\).
\item Prove the kernel of this homomorphism is \(\{\pm I\}\).

\end{enumerate}

This construction actually shows that \(\mathrm{SL}(2, \R)\) is a double cover of \(\mathrm{SO}^+_{2,1}\), the connected component of \(\mathrm{O}_{2,1}\) containing the identity matrix. Or equivalently, \(\mathrm{PSL}(2, \R)=\mathrm{SL}(2, \R)/\{\pm I\}\) is isomorphic to the spin group \(\mathrm{SO}^+(2,1)\). It is an interesting question to show that the orthogonal group \(\mathrm{O}_{2,1}\) has four connected components and identify the geometry of each component.

A similar excercise is to relate $\SL(2, \C)$ to the orthogonal group $\mathrm{O}(3,1)$.
see Artin Algebra Chapter 9, 4.8

\end{ex}

\begin{ex}
  Let $A$ be the set of all $n\times n$ upper triangular matrices with real entries and all diagonal entries equal to $1$. Find the Lie algebra of $A$ and compute its dimension. 
\end{ex}

\begin{ex}
  Show that the intersection of symplectic group $\mathrm{Sp}(2n, \R)$ and orthogonal group $\mathrm{O}(2n, \R)$ in $\GL(2n, \R)$ is isomorphic to the unitary group $\mathrm{U}(n)$. Here we use the embedding of these groups into $\GL(2n, \R)$ as Section Examples of Lie groups and Lie algebras.
\end{ex}

\begin{ex}
  Prove the Jacobi identity of Lie algebra $\mathfrak{gl}(n, \R)=M_n(\R)$ using the properties of the matrix commutator. Here $$[A, B]=AB-BA.$$
\end{ex}








\subsection{Optional excercises}

\begin{ex}
  Prove that the Lie algebra of the symplectic group $\mathrm{Sp}(2n, \R)$ is
  $$\mathfrak{sp}(2n, \R)=\{A\in M_{2n}(\R): A^T J + J A=0\},$$
  where
$$J=\begin{pmatrix}\mathrm{O}_n & \mathrm{I}_n\\
-\mathrm{I}_n & \mathrm{O}_n
\end{pmatrix}.$$
  
\end{ex}

\begin{ex}
  Prove the second and the third order term in the Baker-Campbell-Hausdorff formula when $X, Y$ are elements of $M_n(\R)$. That is, prove
  $$\exp(tX) \exp(tY) = \exp\left(t(X+Y)+\frac{1}{2}t^2[X, Y]+\frac{1}{12}t^3([X,[X, Y]]+[Y,[Y, X]])+\cdots\right)$$
\end{ex}

You may use the following expansion of logarithm:
$$\log(I + A) = A - \frac{1}{2}A^2 + \frac{1}{3}A^3 - \cdots$$

\end{document}