\documentclass{article}
%\usepackage[UTF8]{ctex}

\usepackage{tikz-cd}

\usepackage{amsmath}
\usepackage{amssymb}
\usepackage{amsthm}
\newtheorem{ex}{Problem}

\newcommand{\C}{\mathbb{C}}
\newcommand{\Z}{\mathbb{Z}}
\newcommand{\R}{\mathbb{R}}
\newcommand{\Q}{\mathbb{Q}}
\newcommand{\F}{\mathbb{F}}

\newcommand{\tr}{\operatorname{tr}}
\newcommand{\GL}{\operatorname{GL}}
\newcommand{\SL}{\operatorname{SL}}
\newcommand{\rank}{\operatorname{rank}}
\newcommand{\im}{\operatorname{im}}
\newcommand{\Aut}{\operatorname{Aut}}

\title{RT\ HW6 \\ For you to practice representation theory, you can submit to the TAs if you want to get feedback.}

\begin{document}

\maketitle      

\begin{ex}
  Let $[G,G]$ be subgroup of $G$ generated by all commutators $[g,h]=ghg^{-1}h^{-1}$ for $g,h\in G$. 
  \begin{enumerate}
    \item Show that $[G,G]$ is a normal subgroup of $G$.
    \item Show that $G/[G,G]$ is abelian.
    \item Show that if $H$ is abelian and $\phi\colon G \to H$ is a group homomorphism, then $[G,G]\subset \ker \phi$.
    \item Prove that there is an one-to-one correspondence between the one-dimensional representations of $G$ and the irreducible representations of $G/[G,G]$.
  \end{enumerate}
\end{ex}

\begin{ex}
Let $\SL(2,\R)$ be the group of $2\times 2$ real matrices with determinant $1$ and $W$ be the real vector space of $2\times 2$ real matrices with trace $0$.
\begin{enumerate}
  \item Show that $W$ is a real vector space of dimension $3$.
  \item Show that $W$ is a $\SL(2,\R)$-representation by defining the action of $\SL(2,\R)$ on $W$ by
  \[  A\cdot X = AXA^{-1} \]
  for $A\in \SL(2,\R)$ and $X\in W$.
  \item Show that the bilinear form on $W$ \[
  \langle X,Y \rangle = \tr(XY)
  \]
  is $\SL(2,\R)$-invariant.
  \item Find the signature $(p,q)$ of the bilinear form $\langle \cdot, \cdot \rangle$.
  \item (Optional) Let $O(p,q)$ be the group of linear transformations on $W$ that preserve the bilinear form $\langle \cdot, \cdot \rangle$. Find the image and kernel of the representation $\SL(2,\R) \to O(p,q)$.
\end{enumerate}
\end{ex}

\begin{ex}
  From class, we know that the character $\chi_{reg}$ of regular representation satisfies $\chi_{reg}(g)=0$ if $g\neq e$. There is an inverse of this proposition. Let $\chi$ be a character of $G$ and satisfies $\chi(g)=0$ if $g\neq e$. Prove that the corresponding representation is the direct sum of several copies of regular representation, i.e. $\rho\cong \rho_{reg}\oplus \rho_{reg}\cdots \oplus \rho_{reg}$ for some integer $n$. 
\end{ex}


\end{document}