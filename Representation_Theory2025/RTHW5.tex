\documentclass{article}
%\usepackage[UTF8]{ctex}

\usepackage{tikz-cd}

\usepackage{amsmath}
\usepackage{amssymb}
\usepackage{amsthm}
\newtheorem{ex}{Problem}

\newcommand{\C}{\mathbb{C}}
\newcommand{\Z}{\mathbb{Z}}
\newcommand{\R}{\mathbb{R}}
\newcommand{\Q}{\mathbb{Q}}
\newcommand{\F}{\mathbb{F}}

\newcommand{\tr}{\operatorname{tr}}
\newcommand{\GL}{\operatorname{GL}}
\newcommand{\rank}{\operatorname{rank}}
\newcommand{\im}{\operatorname{im}}
\newcommand{\Aut}{\operatorname{Aut}}

\title{RT\ HW5 \\ Due 8/12, please submit your solutions to the TAs in tutorial session}

\begin{document}

\maketitle      

\begin{ex}
  Let $G_1$ and $G_2$ be two groups and $V_i$ is an representation of $G_i$ over $\C$ for $i=1,2$. Define the representation $V_1 \otimes V_2$ of $G_1 \times G_2$ by
  \[
  (g_1, g_2) \cdot (v_1\otimes v_2) = (g_1 \cdot v_1)\otimes(g_2 \cdot v_2)
  \]
  for $g_i \in G_i$ and $v_i \in V_i$.
  Prove that $V_1 \otimes V_2$ is irreducible if and only if both $V_1$ and $V_2$ are irreducible. How do you express the character of $V_1 \otimes V_2$ in terms of the characters of $V_1$ and $V_2$?
\end{ex}

\begin{ex}
  Let $V$ be an irreducible representation of a finite group $G$ over $\C$. Assume $V$ is not the trivial representation. Prove that for any $v\in V$, we have $\sum_{g \in G} g \cdot v = 0$. (For cyclic groups, this is known to be an identity of roots of unity.)
\end{ex}

\begin{ex}
  Let $G$ be a finite group and operates on a finite set $X$. Let $V$ be the vector space over $\C$ with basis $X$, i.e., $V = \C^X$. Then the $G$-action on $X$ induces a linear representation of $G$ on $V$. Show that number of orbits of $G$ on $X$ is equal to multiplicity of the trivial representation in the irreducible decomposition of $V$.
\end{ex}

\begin{ex}
Prove the following combinatorial identity by characters
\[\sum_{\sigma\in S_n}(\text{number of elements in }[n]\text{ fixed by }\sigma)^2=2(n!)^2\]
\begin{enumerate}
  \item Let $V$ be the representation of $S_n$ on $\C^n$ induced by the action of $S_n$ on standard basis $e_1, \cdots, e_n$ of $\C^n$. Show that the character $\chi_V$ of $V$ is given by
  \[\chi_V(\sigma) = \text{number of elements in }[n]\text{ fixed by }\sigma\]
  for $\sigma \in S_n$.
  \item In class, we showed that the trivial representation of $S_n$ is contained in the representation $V$ given by subspace spanned by $e_1 + e_2 + \cdots + e_n$. The $G$-invariant compliment $W$ is defined by $W = \{v=(v_1\cdots v_n)^T \in \C^n \mid \sum_{i=1}^n v_i = 0\}$. Show that $W$ is irreducible by the following method. Let $V^\prime$ be a nonzero $G$-invariant subspace of $W$ and $v\in V^\prime$ be a non-zero vector. If any two component $v_i$ and $v_j$ of $v$ are not equal, then use a permutation $\sigma \in S_n$ to prove that $e_i-e_j$ is also in $V^\prime$. Then prove that $V^\prime$ must be all of $W$.
  \item Use character of $V$ to prove the combinatorial identity stated in the beginning.
  
\end{enumerate}
\end{ex}







\end{document}