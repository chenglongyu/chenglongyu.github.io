\documentclass[12pt]{article}
\usepackage{amsmath,amsfonts,amsthm,amstext,amssymb,fullpage,framed,graphicx,color}
\usepackage{fullpage}
\usepackage{tikz}\usetikzlibrary{calc}
\usetikzlibrary{quotes,angles}
\pagestyle{empty}


\newcommand\ff{\mathcal{F}}
\newcommand\qq{\mathbb{Q}}
\newcommand\nn{\mathbb{N}}
\newcommand\cc{\mathbb{C}}
\newcommand\lb{\mathcal{L}}
\newcommand\dd{\mathcal{D}}
\newcommand\zz{\mathbb{Z}}
\newcommand\F{\mathbb{F}}
\newcommand\R{\mathbb{R}}

\newcommand\oo{\mathcal{O}}
\newcommand\Hom{{\rm Hom}}
\newcommand\shom{\mathcal{H}{\rm om}}
\newcommand\Ext{{\rm Ext}}
\newcommand\rr{{\mathbb{R}}}
\newcommand\pt{{\mathbb{P}^2}}
\newcommand\po{{\mathbb{P}^1}}
\newcommand\aaa{{\mathbb{A}}}

\DeclareMathOperator{\GL}{GL}
\DeclareMathOperator{\Or}{O}
\DeclareMathOperator{\Aut}{Aut}




\newtheorem{thm}{Theorem}[section]
\newtheorem{lem}[thm]{Lemma}
\newtheorem{prop}[thm]{Proposition}
\newtheorem{cor}[thm]{Corollary}
\theoremstyle{definition}
\newtheorem{defn}[thm]{Definition}
\theoremstyle{remark}
\newtheorem{rmk}[thm]{Remark}
\newtheorem{ex}{Ex}[section]

\begin{document}
\centerline{\bf\large Representation theory projects}
\centerline{}



%\section*{Reading}


%#1 
%\item List all parking functions for $n=3$. 

%#2
%\item Prove $${\displaystyle {m+n \choose r}=\sum _{k=0}^{r}{m \choose k}{n \choose r-k}}$$

%#3
%\item Prove the following propositions for chromatic polynomial $\chi_G(q)$ for graph $G=(V,E)$. Let $n=|V|$ be the number of vertices.
%\begin{enumerate}
%\item $\deg \chi_G(q)= n$
%\item $\chi_G(q)$ can be written in the form $$\chi_G(q)=q(a_0(G)q^{n-1}-a_1(G)q^{n-2}+a_2(G)q^{n-3}+\cdots+ (-1)^{n-1}a_{n-1}(G))$$ with $a_i(G)\in \zz$ and $a_i(G)\geq 0$.
%\item If $G$ is connected, then $a_i(G)>0$.
%\end{enumerate}
%#6




\section{Project 1 Groups: generators and relations}
In this project, we will study groups generated by reflections. 

\subsection{Symmetric group}
Let $S_n$ be the symmetric group on $n$ elements. In homework, you studied the generators of $S_n$ as transpositions, $s_i=(i,i+1)$ and their relations. More precisely, 
\begin{enumerate}
  \item Show that every element in $S_n$ can be written as a product of elements in $\{s_1, s_2, \ldots, s_{n-1}\}$.
  \item Prove that the elements satisfy the following equalities:
  \begin{align*}  
    s_i s_j &= s_j s_i \quad \text{if } |i - j| > 1, \\
    s_i s_{i+1} s_i&= s_{i+1} s_i s_{i+1}, \\
    s_i^2 &= e \quad \text{for all } i.
  \end{align*}

  \item For each permutation $\sigma\in S_n$, define the number of inversions of $\sigma$ as the number of pairs $(i,j)$ such that $i<j$ and $\sigma(i)>\sigma(j)$. Show that the number of inversions is equal to the minimal number of $s_i$ used to express $\sigma$ as their product (counted with multiplicities). For example, the number for $\sigma=s_1s_2s_1$ is three.

  \item Show that the relations above are sufficient to determine the group $S_n$, i.e., for any two different expressions of the same element in $S_n$, they can be transformed into each other using the relations above.
  
  \item Find the minimal sets of transpositions $s_{ij}=(i,j)$ that generate $S_n$. How many are there?
  \item For each set of such generators, find the relations to determine the group $S_n$ and prove your conclusion.
\end{enumerate}

\subsection{Dihedral group}
Let $D_n$ be the symmetry group of a regular $n$-gon. 
\begin{enumerate}
    \item Find the minimal sets of reflections that generate $D_n$.
    \item How many such sets are there?
    \item Find the relations for each set of generators and prove your conclusion.
\end{enumerate}

\subsection{Symmetry group of Platonic solids}
Let $G$ be the symmetry group of a Platonic solid.
\begin{enumerate}
    \item Find the minimal sets of reflections that generate $G$.
    \item How many such sets are there?
    \item Find the relations for one set of generators and prove your conclusion.
\end{enumerate}

\subsection{Symmetry group of tiling patterns}
Let $T$ be the tiling of the plane by equilateral triangles, and $G$ be the symmetry group of such a pattern.
\begin{enumerate}
    \item Find one set of reflections that generate $G$.
    \item Find the relations for the generators you choose and prove your conclusion.
    \item Can you generalize your result to other regular tiling patterns, such as square or hexagonal tiling?
    \item Choose one set of reflections that generate $G$. For each element $\sigma$, find the number of reflections used to express $\sigma$ as their product.
\end{enumerate}




\section{Project 2 Groups and bilinear forms}
In this project, we will study groups arising from certain shapes of hexagons. Consider convex hexagons $P$ with inner angles $2\pi\over 3$. Let the lengths of the sides be $a_1, \cdots, a_6$ in the counterclockwise orientation, and form a vector $a=(a_1, \cdots, a_6)\in \R^6$. 
\begin{enumerate}
\item Find the vector space $W$ spanned by all such $a$.
\item Consider the group $G$ generated by linear transformations $L_i, \quad i=1, \cdots, 6$ of $W$ in the form of $$L_i\colon a_i\mapsto -a_i, a_{i-1}\mapsto a_{i-1}+2a_i, a_{i+1}\mapsto a_{i+1}+2a_i, a_j\mapsto a_j, \text{for other }j$$
  Check that $W$ is invariant under the action of $G$ and that $G$ is an infinite group. Here indices are taken modulo $6$.
\item Find a bilinear form $B\colon W\times W\to \R$ that group $G$ preserves.
\item Is this bilinear form unique? If not, find all bilinear forms that are preserved by the group $G$.

\item Let $C$ be the vectors $v$ in $W$ such that $B(v, v)> 0$. Denote by $F$ the set of vectors $a$ from all such hexagons. Show that $F$ is a subset of $C$. Describe the union of orbits of $F$ under the action of $G$. 

\item Find finite subgroups of $G$. Can you find a classification of these kinds of finite subgroups?

\item Find the relations for the generators $L_i$ and prove your conclusion. (Hard problem, you may just try to find and guess the relations, proving that they are all the relations is a hard problem.)

\item Consider the shapes of $P$ such that it admits a tiling (decomposition) into regular triangles with unit length. Can you describe the possible vectors $a$ from such $P$ and their $G$-orbits in $W$? What are the possible numbers of triangles used in the tiling and for each $n$, is there a counting formula for $c(n)$ the number of different shapes of hexagons with $n$ triangles? (Hard problem, but you can try to find a few examples and reduce to an arithmetic problem, other shapes may result in a simpler form.)

\item Try the problem with other shapes, for example a quadrilateral, a pentagon with certain inner angles you prefer.

\end{enumerate}



\section{Project 3 McKay conjecture}
In this project, we will verify the McKay conjectures for some groups. Let $G$ be a finite group and $P$ is a Sylow $p$-subgroup of $G$. Denote by $N_G(P)$ the normalizer of $P$ in $G$. The McKay conjecture states that the number of irreducible representations of $G$ with dimension coprime to $p$ is equal to the number of irreducible representations of $N_G(P)$ with dimension coprime to $p$. It was recently proved after a series of works by many mathematicians. 

\subsection{Finite subgroups of $SO(3)$ and $SU(2)$}
\begin{enumerate}
    \item Consider the dihedral group $D_n$. List all the irreducible representations of $D_n$ and their dimensions.
    \item Give the classification of Sylow $p$-subgroups of $D_n$ and their normalizers.
    \item For $p=2$, verify the McKay conjecture for $D_n$.
    \item For general prime number $p$, verify the McKay conjecture for $D_n$.
    \item Consider finite subgroups of $SO(3)$ and $SU(2)$. List all the irreducible representations of these groups and their dimensions.
    \item Verify the McKay conjecture for finite subgroups of $SO(3)$ and $SU(2)$.
\end{enumerate}

\subsection{Groups of order $pq$ and $p^2q$}
Assume $p$ and $q$ are two distinct prime numbers.
\begin{enumerate}
    \item Let $G$ be a group of order $pq$. List the possible isomorphism classes of $G$.
    \item For each isomorphism class of $G$, list all the irreducible representations of $G$ and their dimensions.
    \item Give the classification of Sylow $p$-subgroups of $G$ and their normalizers.
    \item Verify the McKay conjecture for $G$.
    \item Generalize to groups of order $p^2q$.
\end{enumerate}

\subsection{Groups of other orders}
Can you generalize the McKay conjecture to groups of other types discussed in class? For example, for $\GL(2, \F_p)$ and its Sylow $p$-subgroups. 


\section{Project 4 McKay graph for finite groups}
In this project, we will explore the McKay graph for other finite groups not discussed in class. Let $V$ be a fixed representation of finite group $G$. Consider all the isomorphism classes of irreducible representations $V_i$ of $G$. View all $V_i$ as the vertices, and if $V_j$ appears in the irreducible decomposition of $V\otimes V_i$ with multiplicity $n_{ij}$, then draw $n_{ij}$ edges from $V_i$ to $V_j$. The resulting graph is called the McKay graph of $G$ with respect to $V$.

\subsection{Symmetric group $S_n$}
\begin{enumerate}
\item Consider $S_n$ and the standard representation $V$ of $S_n$ on $\cc^n$. Construct all irreducible representations of $S_3$ and $S_4$. 

\item Draw the McKay graph for $S_3$ and $S_4$ with respect to the standard representation $V$.

\item How does the McKay graph change when we consider other representations $V$ of $S_3, S_4$? 

\end{enumerate}

\subsection{Subgroups in $SO(3)$}
\begin{enumerate}
    \item Consider all finite subgroups in $SO(3)$. Construct all irreducible representations of $G$.
    \item Let $V$ be the standard representation of $SO(3)$ on $\rr^3$ and view it as a representation on $\cc^3$. Draw the McKay graph for each finite subgroup in $SO(3)$ with respect to the standard representation $V$.
    \item How does the McKay graph change when we consider other representations $V$ of $G$? 
\end{enumerate}

\subsection{Finite subgroups in $U(2)$}
\begin{enumerate}
    \item Classify all finite subgroups in $U(2)$. Construct all irreducible representations of $G$.
    \item Let $V$ be the standard representation of $U(2)$ on $\cc^2$. Draw the McKay graph for each finite subgroup in $U(2)$ with respect to the standard representation $V$.
    \item How does the McKay graph change when we consider other representations $V$ of $G$? 
\end{enumerate}

\subsection{Other finite groups}
You can also try a similar problem for finite groups of order $pq$ or $p^2q$ for distinct primes $p$ and $q$. Make your own choice of $V$.

\end{document}































